\documentclass[12pt, a4paper]{extarticle}
\pagenumbering{gobble}
\usepackage{geometry}
 \geometry{
 left=20mm,
 right=20mm,
 top=20mm,
 bottom=20mm,
 }

\usepackage{titlesec}
\titleformat{\section}[block]{\Large\bfseries\filcenter}{}{1em}{}
\titleformat{\subsection}[block]{\large\bfseries\filright}{}{1em}{}
\usepackage[utf8]{inputenc}
\usepackage[english, russian]{babel}

\usepackage[normalem]{ulem}
\usepackage{soul}
\usepackage{verbatim}
\usepackage{tabularx}
\usepackage{tabulary}
\usepackage{hyperref}
\usepackage{setspace}
\usepackage{amssymb,amsfonts,amsmath,cite,enumerate,float,indentfirst}
\usepackage{mathrsfs}
\usepackage{comment}
\usepackage{verbatim}
\usepackage{mathtools}
\usepackage{caption}
\usepackage{wrapfig}
\usepackage{mathtext} %cyrillic text in math mode, not recommended
\usepackage{amsthm}

\hypersetup{				
	unicode=true,           
	pdfkeywords={keyword1} {key2} {key3},
	colorlinks=true, 
	urlcolor=red
}

\captionsetup[table]{justification=centering, singlelinecheck=false, font=large}
\makeatletter
\renewcommand{\fnum@table}{}
\makeatother

\setlength{\parindent}{1.3em}
\setlength{\parskip}{10pt}
\usepackage{setspace}
\usepackage{enumitem}
\setlist[itemize]{topsep=0pt,partopsep=1ex,parsep=1ex}
\begin{document}
	\section{Дыскрэтная матэматыка}
	Практычныя заняткі на беларускай мове для 9-11 груп першага курса. 	\subsection{Пары:}
	\begin{itemize}
		\item Рашэнне задач (максімум 13 пар); \\[6pt]
		\textbf{Q:} Як гэта будзе адбывацца? \\
		\textbf{A:} Я буду скідваць (хутчэй за ўсё ў спецыяльны чацік) файл з задачамі на дзвюх мовах, частка з якіх будзе прарашана на занятку. Выкарыстанне беларускай мовы ўхваляецца, але неабавязкова. Для лепшага пагружэння ў беларускую мову заведзены (у працэсе) слоўнікі невідавочных тэрмінаў і іншых слоў.
		\item 4 кантрольныя (2-4 пары); \\[6pt]
		\textbf{Q:} Чым можна карыстацца? \\
		\textbf{A:} Любой папяровай крыніцай, акрамя прарашанай ужо кантрольнай. .\noindent\rule{0.5cm}{0.4pt}. \\[6pt]
		\textbf{Q:} Што будзе, калі спаліцца за спісваннем? \\
		\textbf{A:} Згаранне абсалютна ўсіх бонусаў незалежна ад таго, калі яны былі зароблены. 
		\item Сістэма кантроля версій git для імбрыкаў (1 пара, па жаданні); \\[6pt]
		\textbf{Q:} Навошта мне git? \\
		\textbf{A:} З дапамогай git можна пісаць любы код як белы чалавек.
		\item Кампутарная вёрстка з дапамогай \LaTeX~для імбрыкаў (1 пара для ўсіх падгруп, па жаданні); \\[6pt]
		\textbf{Q:} Навошта мне \LaTeX? \\
		\textbf{A:} Каб перасесці з іглы Microsoft Office на прыгожыя pdf-кі без попаболю.
		\item Дадатковыя заняткі для тых, каму цяжкавата (колькі спатрэбіцца).
	\end{itemize}

	\subsection{Кантроль ведаў:}
	4 адзнакі за кантрольныя, якія я дасылаю Алегу Іванавічу, які ставіць залік. Кожная адзнака складаецца з трох частак: 
	\begin{itemize}
		\item Сама кантрольная;
		\item <<бал>> за праверку работы аднагрупніка($\in [-1, 1]$); \\[6pt]
		\textbf{Q:} Навошта правяраць адзін аднаго? \\
		\textbf{A:} Па-першае, гэта не самы бессэнсоўны навык, па-другое, калі пашанцуе, па рабоце аднагрупніка можна разабраць заданні, калі сам не справіўся, па-трэцяе, мне будзе прасцей, што пазітыўна скажацца на ўсім астатнім, што ўваходзіць у курс.
		\item Бонус за <<лабу>> (далей без двукоссяў).
	\end{itemize}
	
	\subsection{Q: Што яшчэ за лаба?}
	{\large A:} Аформленыя задачы аднаго з практычных заняткаў, які быў перад кантрольнай. <<Выдаецца>> на тым самым занятку, гэта значыць, што трэба прыйсці на пару (адзіны момант, дзе мяне цікавяць наведванні). Бонус да кантрольнай залежыць ад наступных умоў:
	\begin{itemize}
		\item Колькі ўсяго часу было на афармленне (колькасць тыдняў паміж заняткам і кантрольнай)?
		\item Наколькі складаныя задачы?
		\item Ці выкарыстоўваўся \LaTeX, і наколькі якасным атрымаўся выніковы дакумент, калі так?
		\item Наколькі складаней аформіць pdf, чым напісаць ад рукі?
		\item Ці выкарыстоўваўся git? Калі так, то самастойна ці ў групе з кімсьці? Ці ёсць сэнс з выкарыстання git, якое атрымалася?
		\item Колькі атрымалася рашыць і рашыць правільна, а таксама самастойна? Наколькі граматна і падрабязна аформлена рашэнне (занадта падрабязна не трэба)? \\[6pt]
		\textbf{Q:} Як будзе карацца плагіят? \\
		\textbf{A:} Жэстачайшэ))) За $\approx$ лабы адзнака падзеліцца пароўну на ўсіх. Лепш шчыра сказаць, што рабілі разам. 
		\item Нешта яшчэ, што я ўпусціў.
	\end{itemize}

	Спадзяюся, за бліжэйшы час я даведаюся, калі будуць кантрольныя, тады і змагу прыблізна ацаніць усе лабы. 
	
	\textbf{Раю пакрыць усе заняткі, каб потым можна было па лабах рыхтавацца да экзамену.}
	
	\textit{Гэты файл будзе папаўняцца адказамі на іншыя важныя пытанні, выпраўляцца, калі нешта з заўваг і прапаноў ажыццявіцца, усё гэта (здаецца, ананімна) пісаць \href{https://docs.google.com/document/d/1La38_lqT7PtdRVmlvKIVh_7rIWvHqPBFG7yrWFoKkFM/edit?usp=sharing}{сюды}.}
\end{document}