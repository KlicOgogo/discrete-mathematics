\usepackage{geometry}
\usepackage{titlesec}
\usepackage[utf8]{inputenc}
\usepackage[english, russian]{babel}
\usepackage[normalem]{ulem}
\usepackage{soul}
\usepackage{verbatim}
\usepackage{tabularx}
\usepackage{tabulary}
\usepackage{hyperref}
\usepackage{setspace}
\usepackage{amssymb,amsfonts,amsmath,cite,enumerate,float,indentfirst}
\usepackage{mathrsfs}
\usepackage{comment}
\usepackage{verbatim}
\usepackage{mathtools}
\usepackage{caption}
\usepackage{wrapfig}
\usepackage{mathtext} %cyrillic text in math mode, not recommended
\usepackage{amsthm}
\usepackage{graphicx}
\usepackage{setspace}
\usepackage{enumitem}
\usepackage[table]{xcolor}

\geometry{left=20mm, right=10mm, top=20mm, bottom=20mm}

\setlist[itemize]{topsep=0pt,partopsep=1ex,parsep=1ex}

\titleformat{\section}[block]{\Large\bfseries\filcenter}{}{1em}{}
\titleformat{\subsection}[block]{\large\bfseries\filcenter}{}{1em}{}

\hypersetup{				
	unicode=true,           
	pdfkeywords={keyword1} {key2} {key3},
	colorlinks=true, 
	urlcolor=blue,
	linkcolor=black
}

\setlength{\parindent}{1.3em}
\setlength{\parskip}{10pt}
\pagenumbering{gobble}

% figure captions settings
\captionsetup[figure]{labelfont=bf, justification=centering}
\renewcommand\thefigure{\thesection.\arabic{figure}}  
\makeatletter
\renewcommand{\fnum@figure}{Малюнак \thefigure}
\makeatother

% table captions settings
\captionsetup[table]{justification=centering, singlelinecheck=false, font=large}
\makeatletter
\renewcommand{\fnum@table}{}
\makeatother
\renewcommand\thetable{\thesection.\arabic{table}}  
\makeatletter
\renewcommand{\fnum@table}{Табліца \thetable}
\makeatother

\newcommand{\formQA}[2]{%
	\noindent \textbf{Q:} #1 \\
	\textbf{A:} #2
}


\begin{document}
	\section{Дыскрэтная матэматыка}
	Практычныя заняткі на беларускай мове для 9-11 груп першага курса.
	
	\subsection{Пары}
	\begin{itemize}
		\item Рашэнне задач (колькі спатрэбіцца, прыблізна 13 пар); \\[6pt]
		\formQA{Як гэта будзе адбывацца?}
		{Я буду скідваць файл з задачамі на дзвюх мовах, частка з якіх будзе прарашана на занятку. Выкарыстанне беларускай мовы ўхваляецца, але неабавязкова. Для лепшага пагружэння ў беларускую мову заведзены (у працэсе) слоўнікі невідавочных тэрмінаў і іншых слоў.}
		\item 4 кантрольныя (2-4 пары). Хутчэй за ўсё некаторыя з іх будуць не на цэлую пару; \\[6pt]
		\formQA{Чым можна карыстацца?}
		{Любой папяровай крыніцай, акрамя прарашанай ужо кантрольнай. .\noindent\rule{0.5cm}{0.4pt}. \\[6pt]}
		\formQA{Што будзе, калі спаліцца за спісваннем?}
		{Згаранне абсалютна ўсіх бонусаў незалежна ад таго, калі яны былі зароблены.}
		\item Сістэма кантроля версій git для імбрыкаў (1 пара, па жаданні); \\[6pt]
		\formQA{Навошта мне git?}
		{З дапамогай git можна пісаць любы код як белы чалавек.}
		\item Кампутарная вёрстка з дапамогай \LaTeX~для імбрыкаў (1 пара для ўсіх падгруп, па жаданні); \\[6pt]
		\formQA{Навошта мне \LaTeX?}
		{Каб перасесці з іглы Microsoft Office на прыгожыя pdf-кі без попаболю.}
		\item Дадатковыя заняткі для тых, каму цяжкавата (колькі спатрэбіцца).
	\end{itemize}

	\subsection{Кантроль ведаў}
	Адзнакі па кантрольных, якія я дасылаю Алегу Іванавічу, каб потым яны ўлічваліся пры падліку рэйтынгавай адзнакі на экзамен. Яшчэ яны нейкім чынам будуць уплываць на залік у гэтым семестры. Кожная адзнака складаецца з трох частак:
	\begin{itemize}
		\item Сама кантрольная;
		\item <<бал>> за праверку работы аднагрупніка($\in [-1, 1]$); \\[6pt]
		\formQA{Навошта правяраць адзін аднаго?}
		{Па-першае, гэта не самы бессэнсоўны навык, па-другое, калі пашанцуе, па рабоце аднагрупніка можна разабраць заданні, калі сам не справіўся, па-трэцяе, мне будзе прасцей, што пазітыўна скажацца на ўсім астатнім, што ўваходзіць у курс.}
		\item Бонус за <<лабу>> (далей без двукоссяў).
	\end{itemize}
	\formQA{Па якой формуле вылічваецца адзнака?}
	{Формулы няма :) Зроблена так, таму што яе, па-першае, складана вывесці і, хутчэй за ўсё, варта пад вынікі падганяць, а, па-другое, так павялічваецца шанец, што намаганні будуць накіраваны правільна (не на абход сістэмы). Дакладна вядома, што калі крута напісаць кантрольную і адэкватна праверыць аднагрупніка, усё будзе добра.}
	\subsection{Q: Што яшчэ за лаба?}
	{\large A:} Аформленыя задачы аднаго з практычных заняткаў, які быў перад кантрольнай. Правіла, што трэба прыйсці на занятак, дзе разбіралася паперка, тупое, таму яно скасоўваецца: абмежаванняў няма. Бонус да кантрольнай залежыць ад наступных умоў:
	\begin{itemize}
		\item Колькі ўсяго часу было на афармленне (колькасць тыдняў паміж заняткам і кантрольнай)?
		\item Наколькі складаныя задачы?
		\item Ці выкарыстоўваўся \LaTeX, і наколькі якасным атрымаўся выніковы дакумент, калі так?
		\item Наколькі складаней аформіць pdf, чым напісаць ад рукі?
		\item Ці выкарыстоўваўся git? Калі так, то самастойна ці ў групе з кімсьці? Ці ёсць сэнс з выкарыстання git, якое атрымалася?
		\item Колькі атрымалася рашыць і рашыць правільна, а таксама самастойна? Наколькі граматна і падрабязна аформлена рашэнне (занадта падрабязна не трэба)? \\[6pt]
		\formQA{Як будзе карацца плагіят?}
		{Жэстачайшэ))) За $\approx$ лабы адзнака падзеліцца пароўну на ўсіх. Лепш шчыра сказаць, што рабілі разам.}
	\end{itemize}

	\textbf{Раю пакрыць усе заняткі, каб потым можна было па лабах рыхтавацца да экзамену.}
	
	\subsection{Залік}

	Дастатковая ўмова заліку: усе кантрольныя напісаны на 4+. Неабходнай умовы няма: калі не будзе кантрольных на 4+, атрымаць залік аўтаматам усё яшчэ мажліва, але ўсё будзе залежыць ад усёй працы па ходу семестра. Спадзяюся, нікому не прыйдзецца нічога здаваць. Але калі да гэтага дойдзе, хутчэй за ўсё трэба будзе рашыць задачу ці пару задач. Буду глядзець, ці ёсць разуменне таго, як што атрымліваецца ў рашэнні. 

	У цэлым Алег Іванавіч мне параіў не адпраўляць лішні раз на пераздачу, але калі нехта будзе зусім дно па ўсіх параметрах, то дасвідулі.

	\textit{Гэты файл будзе папаўняцца адказамі на іншыя важныя пытанні, выпраўляцца, калі нешта з заўваг і прапаноў ажыццявіцца, усё гэта (здаецца, ананімна) пісаць \href{https://docs.google.com/document/d/1La38_lqT7PtdRVmlvKIVh_7rIWvHqPBFG7yrWFoKkFM/edit?usp=sharing}{сюды}.}
\end{document}