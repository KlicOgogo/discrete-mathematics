\usepackage{geometry}
\usepackage{titlesec}
\usepackage[utf8]{inputenc}
\usepackage[english, russian]{babel}
\usepackage[normalem]{ulem}
\usepackage{soul}
\usepackage{verbatim}
\usepackage{tabularx}
\usepackage{tabulary}
\usepackage{hyperref}
\usepackage{setspace}
\usepackage{amssymb,amsfonts,amsmath,cite,enumerate,float,indentfirst}
\usepackage{mathrsfs}
\usepackage{comment}
\usepackage{verbatim}
\usepackage{mathtools}
\usepackage{caption}
\usepackage{wrapfig}
\usepackage{mathtext} %cyrillic text in math mode, not recommended
\usepackage{amsthm}
\usepackage{graphicx}
\usepackage{setspace}
\usepackage{enumitem}
\usepackage[table]{xcolor}

\geometry{left=20mm, right=10mm, top=20mm, bottom=20mm}

\setlist[itemize]{topsep=0pt,partopsep=1ex,parsep=1ex}

\titleformat{\section}[block]{\Large\bfseries\filcenter}{}{1em}{}
\titleformat{\subsection}[block]{\large\bfseries\filcenter}{}{1em}{}

\hypersetup{				
	unicode=true,           
	pdfkeywords={keyword1} {key2} {key3},
	colorlinks=true, 
	urlcolor=blue,
	linkcolor=black
}

\setlength{\parindent}{1.3em}
\setlength{\parskip}{10pt}
\pagenumbering{gobble}

% figure captions settings
\captionsetup[figure]{labelfont=bf, justification=centering}
\renewcommand\thefigure{\thesection.\arabic{figure}}  
\makeatletter
\renewcommand{\fnum@figure}{Малюнак \thefigure}
\makeatother

% table captions settings
\captionsetup[table]{justification=centering, singlelinecheck=false, font=large}
\makeatletter
\renewcommand{\fnum@table}{}
\makeatother
\renewcommand\thetable{\thesection.\arabic{table}}  
\makeatletter
\renewcommand{\fnum@table}{Табліца \thetable}
\makeatother

\newcommand{\formQA}[2]{%
	\noindent \textbf{Q:} #1 \\
	\textbf{A:} #2
}


\renewcommand{\ge}{\geqslant}

\begin{document}
    \section{Залік}
    Гэты док прысвечаны заліку па курсе.
    
    \subsection{Аўтаматы}
    Тыя, хто не мае <<чыстую рэпутацыю>> (забруджваецца спісваннем, якое нельга было не заўважыць, падманам і іншым), здаюць залік у любым выпадку.
    Залік аўтаматам можна атрымаць, калі выконваецца адна з умоў:
    \begin{itemize}
        \item Абедзве адзнакі за кантрольныя $\ge 4$;
        \item Сума адзнак за кантрольныя $\ge 9$;
        \item Другая кантрольная напісана на $\ge 5$ балаў лепш за першую.
    \end{itemize}

    \subsection{Фармат правядзення}
    
    \begin{enumerate}
        \item Колькасць задач, якія трэба правільна рашыць на заліку, вызначаецца сумарнай колькасцю па наступных пунктах:
        \begin{itemize}
            \item Адна задача за тое, што не выконваецца ніводная з дастатковых умоў аўтамата.
            \item $\left\lfloor \dfrac{9 - S}{1.5} \right\rfloor$ задач, дзе $S$ "--- сума балаў па ўсіх кантрольных.
            \item $RD^2$, дзе $RD$ "--- сумарная колькасць рэпутацыйных удараў.
            \item Нулявая задача для ўсіх, якая ідзе першая і якую трэба абавязкова рашыць. У процілеглым выпадку адразу ставіцца незалік.
        \end{itemize}
        \item Пасля праверкі другой кантрольнай работы кожны будзе ведаць сваю колькасць задач. Каб не было непаразуменняў, я ствару табліцы, дзе для кожнага будзе падлічана колькасць задач на залік.
        \item Пасля таго, як нулявая задача была вырашана, далей ідуць <<туры>> з адной задачы. Усе, хто яшчэ не здаў, рашаюць роўна 15 хвілін, я потым хутка правяраю. 
        \item Калі не здаць рашэнне ў час, задача не правяраецца і лічыцца няправільнай. 
        \item Для здачы заліку трэба правільна рашыць столькі задач, колькі было вызначана ў пункце 1, не лічачы нулявую.
        \item Double elimination: калі няправільна рашыць задачы на двух турах, ставіцца незалік. Таксама гэта можна лічыць жоўтай дарожкай перадачы <<Умницы и умники>> зменнай даўжыні.
        \item Задачы будуць брацца тыя, якія разбраліся на практычных занятках (гэта дае мажлівасць добра падрыхтавацца, а мне не траціць лішні час на пошук задач).
        \item У астатнім працуюць правілы, вызначаныя ў апісанніі курса, акрамя аднаго: з усяго, што зараз расцэньваецца як удар па рэпутацыі, аўтамачычна вынікае незалік.
    \end{enumerate}

    \subsection{Дата і час правядзення}
    
    Пакуль дакладна не магу сказаць, але, калі атрымаецца дамовіцца з Алегам Іванавічам, залік будзе прымацца ў чацвер а 8:15 у заліковы тыдзень (6 чэрвеня, я так разумею). Калі трэба будзе праводзіць у той дзень звычайную пару, залік адбудзецца ў суботу.
    
    \subsection{Дадаткова}
    
    \begin{itemize}
        \item У кожнай групе студэнт з найвялікшай сумай балаў мае права вызваліць аднагрупніка ад адной задачы.
        \item На пераздачы будзе тое ж самае, толькі без нулявой задачы. Колькасць задач будзе роўная той колькасці задач, на якой спынілася здача заліку.
        \item У суботу, 1 чэрвеня, будзе дадатковы занятак, прысвечаны падрыхтоўцы. Кожная група выбірае па 3 задачы з пройдзеных тэм, якія будуць разабраны, таксама будзе адна задача ад мяне. Час будзе такі, каб усе групы маглі прыйсці.
    \end{itemize}
\end{document}