\documentclass[12pt, a4paper]{extarticle}
\usepackage{import}

\subimport{../common/}{preamble}
\renewcommand{\ge}{\geqslant}

\begin{document}
    \section{Залік}
    Гэты док прысвечаны заліку па курсе.
    
    \subsection{Аўтаматы}
    Тыя, хто не мае <<чыстую рэпутацыю>> (забруджваецца спісваннем, якое нельга было не заўважыць, падманам і іншым), здаюць залік у любым выпадку.
    Залік аўтаматам можна атрымаць, калі выконваецца адна з умоў:
    \begin{itemize}
        \item Абедзве адзнакі за кантрольныя $\ge 4$;
        \item Сума адзнак за кантрольныя $\ge 9$;
        \item Другая кантрольная напісана на $\ge 5$ балаў лепш за першую.
    \end{itemize}

    \subsection{Фармат правядзення}
    
    \begin{enumerate}
        \item Колькасць задач, якія трэба правільна рашыць на заліку, вызначаецца сумарнай колькасцю па наступных пунктах:
        \begin{itemize}
            \item Адна задача за тое, што не выконваецца ніводная з дастатковых умоў аўтамата.
            \item $\left\lfloor \dfrac{9 - S}{1.5} \right\rfloor$ задач, дзе $S$ "--- сума балаў па ўсіх кантрольных.
            \item $RD^2$, дзе $RD$ "--- сумарная колькасць рэпутацыйных удараў.
            \item Нулявая задача для ўсіх, якая ідзе першая і якую трэба абавязкова рашыць. У процілеглым выпадку адразу ставіцца незалік.
        \end{itemize}
        \item Пасля праверкі другой кантрольнай работы кожны будзе ведаць сваю колькасць задач. Каб не было непаразуменняў, я ствару табліцы, дзе для кожнага будзе падлічана колькасць задач на залік.
        \item Пасля таго, як нулявая задача была вырашана, далей ідуць <<туры>> з адной задачы. Усе, хто яшчэ не здаў, рашаюць роўна 15 хвілін, я потым хутка правяраю. 
        \item На нулявую задачу даецца 10 хвілін. Калі часу не хапае, можна набыць яшчэ 5 хвілін за адну дадатковую задачу на залік, але толькі адзін раз.
        \item Калі не здаць рашэнне ў час, задача не правяраецца і лічыцца няправільнай. 
        \item Для здачы заліку трэба правільна рашыць столькі задач, колькі было вызначана ў пункце 1, не лічачы нулявую.
        \item Double elimination: калі няправільна рашыць задачы на двух турах, ставіцца незалік. Таксама гэта можна лічыць жоўтай дарожкай перадачы <<Умницы и умники>> зменнай даўжыні.
        \item Задачы будуць брацца тыя, якія разбіраліся на практычных занятках (problems\_01-10) і кантрольных (гэта дае мажлівасць добра падрыхтавацца, а мне не траціць лішні час на пошук задач). Задачы кантрольных можна будзе знайсці ў telegram, а таксама ў \href{https://github.com/bsu-docs/discrete-mathematics-problems}{рэпазіторыі} з усімі задачамі па практыцы.
        \item У астатнім працуюць правілы, вызначаныя ў апісанні курса, акрамя аднаго: з усяго, што зараз расцэньваецца як удар па рэпутацыі, аўтамачычна вынікае незалік.
    \end{enumerate}

    \subsection{Дата і час правядзення}
    
    Залік адбудзецца 6 чэрвеня а 8:15 у 607 аўдыторыі. Далей ён перамяшчаецца ў 600в, а калі спатрэбіцца і трэцяя пара, то ў 519. Залік скончыцца а 13:00, і тыя, хто не паспеў дарашаць усе задачы, атрымліваюць незалік. Пары, якія стаць на час правядзення заліку, у выніку адбудуцца, як звычайна.

    \subsection{Дадаткова}
    
    \begin{itemize}
        \item У кожнай групе студэнт з найвялікшай сумай балаў мае права вызваліць аднагрупніка ад адной задачы.
        \item На пераздачы будзе тое ж самае, толькі без нулявой задачы. Колькасць задач будзе роўная той колькасці задач, на якой спынілася здача заліку.
        \item У суботу, 1 чэрвеня, а 8:15 у 607 аўдыторыі будзе дадатковы занятак, прысвечаны падрыхтоўцы. Кожная група выбірае па 3 задачы з пройдзеных тэм, якія будуць разабраны, таксама будзе адна задача ад мяне. Нумары задач трэба мне напісаць у пятніцу, 31 траўня, да 22 вечара.
        \item Пытанне з тым, як працягваць залік, калі далей ідзе іншае пара, рашаецца студэнтамі самастойна. Іншых залікаў не павінна быць, таму можна дамовіцца з ваыкладчыкам ці прагуляць. Дазваляецца перарваць здачу на другой пары, каб прыйсці на трэцюю, але трэба ўлічваць, што залік скончыцца а~13:00.
        \item Фармальна залік адбудзецца ў той жа дзень а 13:00 у 610 аўдыторыі. Усе, хто здаў мне, прыносяць ці перадаюць залікоўкі, каб Алег Іванавіч паставіў залік. Астатнія чакаюць наступны шанец.
    \end{itemize}
\end{document}