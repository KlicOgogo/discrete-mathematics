\documentclass[12pt, a4paper]{extarticle}
\usepackage{import}

\subimport{../../common/}{preamble}

\begin{document}
    \section{Нулявая задача}
    \noindent Даша і Ксюша пайшлі ў валанцёры і атрымалі залік у Алега Іванавіча. Але, калі б яны паўтарылі свае адзнакі за першую кантрольную, то ім бы таксама прыйшлося здаваць залік Андрэю, і яны б мелі на залік адпаведна 5 і 4 задачы. Залік павінен быў адбыцца па схеме double elimination, апісанай у дакуменце setoff/setoff.pdf, пры якой трэба рашыць сваю колькасць задач, не рашыўшы ці рашыўшы няправільна не больш за адну задачу. Колькі максімальна задач магло быць на заліку, які б праводзіўся спецыяльна для іх, і Даша і Ксюша рашалі б з самага пачатку без разрываў? Пад колькасцю задач разумеецца колькасць тураў, дзе тур "--- суцэльная частка заліку, на якой усім удзельнікам даецца адна задача, яна рашаецца, а пасля правяраецца і ацэньваецца альбо станоўча, альбо адмоўна. Пераздача ставіцца пры наяўнасці дзвюх адмоўных адзнак. Фраза <<рашалі без разрываў>> азначае, што Даша і Ксюша рашаюць з першага тура, не прапускаючы ніякі наступны. \\
    {\footnotesize \textcolor{darkgray}{Даша и Ксюша пошли в волонтёры и получили зачёт у Олега Ивановича. Но, если бы они повторили свои оценки за первую контрольную, то им бы тоже пришлось сдавать зачёт Андрею, и они бы имели на зачёт 5 и 4 задачи соответственно. Зачёт должен был состояться по схеме double elimination, описанной в  документе setoff/setoff.pdf, при которой необходимо решить своё количество задач, не решив либо решив неправильно не более, чем одну задачу. Сколько максимально задач могло быть на зачёте, который бы проводился специально для них, и Даша и Ксюша решали б с самого начала без разрывов? Под количеством задач подразумевается количество туров, где тур "--- цельная часть зачёта, на которой всем даётся одна задача, она решается, а потом проверяется и оценивается либо положительно, либо отрицательно. Пересдача ставится при получении двух отрицательных оценок. Фраза <<решали без разрывов>> означает, что Даша и Ксюша решают с первого тура, не пропуская никакой следующий.}}
\end{document}