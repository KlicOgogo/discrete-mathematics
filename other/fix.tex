\documentclass[12pt,twoside]{article}
\usepackage{import}

\subimport{../common/}{preamble}
\begin{document}
    \noindent \textbf{Задача}: рашыць раўнанне
    \begin{equation} \label{eq1}
    \sum _{i=1} ^k x_i = n,\ x_i \ge 1.
    \end{equation}
    
    \noindent \textbf{Рашэнне:} \\
    1) Звесці задачу да задачы выгляду: 
    \begin{equation} \label{eq2}
    \sum _{i=1} ^k x_i = n,\ x_i \ge 0.
    \end{equation}
    Для гэтага трэба зрабіць замену $y_i=x_i-1$. \\
    2) Рашыць атрыманую задачу. Будзем лічыць, што лікі $x_i$ "--- гэта <<карзіны>>, у якія кладуцца адзінкі. Калі мы па $x_i$ размяркуем $n$ адзінак, атрымаецца рашэнне раўнання \eqref{eq2}. Таму колькасць рашэнняў раўнання \eqref{eq2} "--- гэта колькасць спосабаў размеркаваць $n$ элементаў у $k$ адрозных карзін з паўтарэннямі, а гэта роўна $\overline{C_n ^k}$, адпаведна, колькасць рашэнняў раўнання \eqref{eq1} "--- $\overline{C_{n-k} ^k}$.
\end{document}