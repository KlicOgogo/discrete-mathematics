\documentclass[12pt, a4paper]{extarticle}
\usepackage{import}

\subimport{../common/}{preamble}

\begin{document}
    \section{Яшчэ адно апісанне таго, што я назваў лабай}
    Больш бессістэмнасці богу бессістэмнасці!!! Разбалоўкі па задачах не будзе, толькі максімальны бал, калі можна атрымаць за ўсе задачы. Нагадваю, што НІКОМУ, нават мне, невядома, як атрыманы бал будзе пераўтвораны ў выніковую адзнаку. Формулу падліку адзнакі па лабе можна апісаць прыблізна так:
    $$G = (a_i + b_i + c_i)  \cdot  d,$$ дзе $a_i$ "--- каэфіцыент за афармленне <<простых смяротных>>, $a_i = A_i$ ; $b_i$ "--- каэфіцыент за якасць выкарыстання \LaTeX, $b_i \in [0, B_i]$; $c_i$ "--- каэфіцыент за якасць выкарыстання git, $c_i \in [0, C_i]$; $i$ "--- нумар ЗДАДЗЕННАЙ лабы, $d$ "--- сума балаў за правільна зробленыя заданні, $d \in [0, \sum]$, дзе $\sum$ "--- максімальны бал лабы, указаны ў табліцы. Для гэтых каэфіцыентаў справядлівыя наступныя законы:
    \begin{equation*}
        \begin{aligned}
        	a_i + b_i + c_i & \leqslant 1 \\	
        	a_1 = A_1 & = 0.2875 \\
        	b_1 \leqslant B_1 & = 0.3875 \\
        	c_1 \leqslant C_1 & = 0.325 \\
        	A_i + B_i + C_i & = 1 \\
        	b_i \rightarrow B_i & \Rightarrow B_{i+1} \downarrow \downarrow \\
        	c_i \rightarrow C_i & \Rightarrow C_{i+1} \downarrow \downarrow \\
        	b_i \rightarrow 0 & \Rightarrow B_{i+1} \uparrow \uparrow \\
        	c_i \rightarrow 0 & \Rightarrow C_{i+1} \uparrow \uparrow \\
        \end{aligned}
    \end{equation*}
    
    \subsection{Табліца з разбалоўкай:}
    \begin{table}[H]
    	\begin{center}
    		\begin{tabular}{|l|l|l|}
    			\hline
    			\bf №  & $\sum$ & \bf Deadline  \\ \hline
    			1    & 20 & 08-13.04\\ \hline
    			2   & 55 & 08-13.04 \\ \hline
    			3   & 80 & 08-13.04 \\ \hline
    			4   & 75 & 08-13.04 \\ \hline
    			5   & 85 & 08-13.04 \\ \hline
    			7   & 80 & 30.05-03.06 \\ \hline
    			8   & 70 & 30.05-03.06  \\ \hline
    			9   & 75 & 30.05-03.06  \\ \hline
    			10   & 200 & 30.05-03.06  \\ \hline
    		\end{tabular}
    	\end{center}
    \end{table}
    Дэдлайн "--- а 23:59:59 у дзень перад бліжэйшай кантрольнай. Пытанне, калі будуць кантрольныя, па ходу будзе рашацца не загадзя, таму не гарантую, што пра дэдлайны аб'яўлю дастаткова рана. Пастараюся як мага хутчэй абнаўляць слупок з датамі.
    
    \formQA{Што, калі здаць некалькі аформленых заняткаў перад адной кантрольнай?}
    {У вынік пойдзе лепшая з сум балаў, але хутчэй за ўсё мне будзе складана праігнараваць гэта працу.}
    
    Вынікі здадзеных прац будуць дадавацца ў табліцы па групах, якія будуць ляжаць побач з гэтым докам у папцы на github.
\end{document}