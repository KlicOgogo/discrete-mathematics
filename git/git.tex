\documentclass[12pt, a4paper]{extarticle}
\pagenumbering{gobble}
\usepackage{geometry}
 \geometry{
 left=20mm,
 right=20mm,
 top=20mm,
 bottom=20mm,
 }

\usepackage{titlesec}
\titleformat{\section}[block]{\Large\bfseries\filcenter}{}{1em}{}
\titleformat{\subsection}[block]{\large\bfseries\filcenter}{}{1em}{}
\usepackage[utf8]{inputenc}
\usepackage[english, russian]{babel}

\usepackage[normalem]{ulem}
\usepackage{soul}
\usepackage{verbatim}
\usepackage{tabularx}
\usepackage{tabulary}
\usepackage{hyperref}
\usepackage{setspace}
\usepackage{amssymb,amsfonts,amsmath,cite,enumerate,float,indentfirst}
\usepackage{mathrsfs}
\usepackage{comment}
\usepackage{verbatim}
\usepackage{mathtools}
\usepackage{caption}
\usepackage{wrapfig}
\usepackage{mathtext} %cyrillic text in math mode, not recommended
\usepackage{amsthm}
\usepackage{graphicx}

\hypersetup{				
	unicode=true,           
	pdfkeywords={keyword1} {key2} {key3},
	colorlinks=true, 
	urlcolor=blue
}

\captionsetup[table]{justification=centering, singlelinecheck=false, font=large}
\makeatletter
\renewcommand{\fnum@table}{}
\makeatother

\captionsetup[figure]{labelfont=bf, justification=centering}

% figure captions settings
\renewcommand\thefigure{\thesection.\arabic{figure}}  
\makeatletter
\renewcommand{\fnum@figure}{Малюнак \thefigure}
\makeatother

% table captions settings
\renewcommand\thetable{\thesection.\arabic{table}}  
\makeatletter
\renewcommand{\fnum@table}{Табліца \thetable}
\makeatother



\newcommand{\formQA}[2]{%
	\noindent \textbf{Q:} #1 \\
	\textbf{A:} #2
}

\setlength{\parindent}{1.3em}
\setlength{\parskip}{10pt}
\usepackage{setspace}
\usepackage{enumitem}
\setlist[itemize]{topsep=0pt,partopsep=1ex,parsep=1ex}

\graphicspath{{images/}}

\begin{document}

\section{git з github для імбрыкаў}

У гэтым доку будуць інструкцыя па наладжванню акаўнта на \href{https://github.com/}{github.com}, паслядоўнасць крокаў для стварэння там свайго рэпазіторыя, унутры якога можна карыстацца сродкамі git, а таксама базавыя каманды git і іншыя карысныя штукі. Для таго, каб усталяваць git на свой кампукцер, ёсць \href{https://git-scm.com/book/en/v2/Getting-Started-Installing-Git}{старонка} з інструкцыяй для кожнай аперацыйнай сістэмы.

На самай справе можна закрыць гэты док і знайсці нешта ў інтэрнэце, напрыклад \href{https://habr.com/ru/post/125799/}{гэта} (рэкамендую чытаць разам з каментамі), \href{https://www.tutorialspoint.com/git/index.htm}{гэта} ці \href{https://githowto.com}{гэта}. Таксама можна ўзяць і прачытаць \href{https://git-scm.com/book/en/v1/}{кнігу}, у якой усё даступна тлумачыцца.

\subsection{Акаўнт на github.com}

\begin{itemize}

\item Ствараем акаўнт: заходзім на сайт, клікаем на адну з кнопак \textbf{Sign up}, выконваем усе, што просяць.

\item Пасля стварэння акаўнта github.com/<username> будзе дамашняй старонкай.

\item Для бяспечнага карыстання рэкамендуецца наладзіць злучэнне з github па ssh. Ёсць добрая \href{https://help.github.com/en/articles/generating-a-new-ssh-key-and-adding-it-to-the-ssh-agent}{інструкцыя} для ўсіх асноўных аперацыйных сістэм. Каб праверыць, што злучэнне наладжана, трэба выканаць адну каманду ў кансолі, чытаць \href{https://help.github.com/en/articles/testing-your-ssh-connection}{тут}.

\end{itemize}

\subsection{git lfs}

Адны класныя дзядзечкі прыдумалі сродак, каб не мяшаць код і здаровыя і не вельмі бінарныя файлы ў адну кучу (то-бок можна падцягнуць з сервераў толькі код калі патрэбы ў астатнім няма ці трэба пампаваць вельмі шмат, а інтэрнэт не дазваляе па адной з прычын). Вельмі проста ўсталяваць, вельмі проста карыстацца, абы толькі не лагала. .-.

Інструкцыя па ўсталёўцы і наладжванні \href{https://github.com/git-lfs/git-lfs}{тут}. Толькі рэкамендую каманду \textbf{git lfs install -\hspace{1pt}-skip-smudge} замест \textbf{git lfs install}, каб можна было самастойна кіраваць зацягваннем вялікіх файлаў (\textbf{git lfs pull}). 

\subsection{Стварэнне і налады рэпазіторыя}

Каб стварыць свой рэпазіторый на github.com, трэба зайсці на старонку свайго профіля і клікнуць <<+>> у правым верхнім куце (неяк няёмка прыкладваць скрыншот), абраць <<New repository>> і наладзіць. Рэкамендую дадаваць README і .gitignore. README пішацца на \href{https://en.wikipedia.org/wiki/Markdown}{Markdown} і дазваляе эфектна апісаць, для чаго рэпазіторый быў створаны і як карыстацца тым, што там знаходзіцца. Таксама github дазваляе з гэтага файла стварыць прыгожую дамашнюю старонку нібыта сайта вашага рэпазіторыя. .gitignore "--- гэта файл, у якім апісваецца ўсё, што не будзе загружацца на копію рэпазіторыя на серверы. Напрыклад, калі пісаць праект на старой Microsoft Visual Studio (не магу сцвярджаць пра новыя версіі), кожны праект змяшчае, апрача файлаў з кодам, кучу аўтаматычна створаных іншых файлаў (.sln), якія нікуды не трэба заліваць. 

Калі трэба працаваць не са сваім кодам, а з чужым, які мае свой рэпазіторый на github, можна стварыць сваю копію і працаваць з ёй, для гэтага трэба зайсці на рэпазіторый і тыкнуць на <<fork>> (малюнак \ref{fork}).

\begin{figure}[H]

\begin{center}
\includegraphics[width=\textwidth]{fork.png}
\end{center}

\caption{Копія чужога рэпазіторыя} \label{fork}
\end{figure}

Пасля таго, як рэпазіторый быў створаны, трёба наладзіць яго ў сябе. Па-першае, трэба зацягнуць копію з github, для гэтага трэба адкрыць кансоль (git bash) і выканаць каманду \textbf{git clone <repo url>}, дзе <repo url> "--- спасылка, якую выдае github, калі на старонцы рэпкі клікнуць <<Clone or download>> (малюнак \ref{clone}).

\begin{figure}[H]

\begin{center}
	\includegraphics[width=\textwidth]{clone.png}
\end{center}

\caption{Спасылка для git clone} \label{clone}
\end{figure}

Пасля таго, як копія з'явілася на вашым кампе, трэба прапісаць у наладах электронны адрас і імя, якія будуць выкарыстоўвацца для подпісу камітаў (што гэта такое, будзе потым). Гэта можна зрабіць глабальна альбо ў межах рэпазіторыя, у якім вы знаходзіцеся (каманда \textbf{pwd} скажа, у якой папцы вы знаходзіцеся ў тэрмінале (git bash)). Усе патрэбныя каманды ёсць \href{https://git-scm.com/book/en/v2/Customizing-Git-Git-Configuration}{тут} (каб наладзіць лакальна, трэба прыбраць \textbf{-\hspace{1pt}-global}). Яшчэ можна чытануць \href{https://help.github.com/en/articles/about-commit-email-addresses}{гэта}.

Пасля выканання гэтых крокаў можна наладжваць git lfs. 

\subsection{Аддаленыя серверы}

Без разумення прынцыпаў узаемадзеяння лакальнай копіі і копіі на github будзе складана свядома працаваць з git. Лепшы \href{https://git-scm.com/book/en/v2/Git-Basics-Working-with-Remotes}{артыкул} (ёсць рускамоўны варыянт, пераклад на беларускую мову абяцаюць), каб разабрацца. Няма нічога лепш, чым рукамі памацаць, галоўнае, нічога сабе толькі не паламаць (патрэніравацца на нейкім тэставым рэпазіторыі), можна завесці два экзэмпляры, каб разабрацца з upstream і origin. 

\subsection{Праца з git на базавым узроўні}

Не думаю, што тут я магу справіцца на ўзроўні гатовых прыкладаў, таму далей на кожны крок будзе спасылка на кнігу.

\begin{enumerate}

\item \href{https://git-scm.com/book/en/v1/Getting-Started-Git-Basics}{Разбіраемся}, што такое каміт.

\item \href{https://git-scm.com/book/en/v1/Git-Basics-Recording-Changes-to-the-Repository}{Вучымся} працаваць са зменамі ў рэпазіторыі і захоўваць іх.

\item \href{https://git-scm.com/book/en/v1/Git-Branching}{Засвойваем} моцны інструмент git, які дазваляе карэктна працаваць з рэпазіторыем. Калі ваш рэпазіторый змяняеце толькі вы, можна не замарачвацца і працаваць толькі з галіной <<master>>, але ў іншых выпадках змяняць <<master>> рэкамендуецца праз \href{https://help.github.com/en/articles/about-pull-requests}{pull request}, а тут ужо складана (нельга) без выкарыстоўвання галін.

\item Для вывучэння прынцыпаў git branching ёсць яшчэ такая \href{https://learngitbranching.js.org}{прыкалюха}.

\item Заўсёды ёсць каманда \textbf{git help}. Таксама можна выклікаць дапамогу для каманд git.

\item Калі не хочацца нічога разумець, а проста рабіць, дастаткова ведаць наступныя каманды: <<add>>, <<diff>>, <<reset>>, <<commit>>, <<push>>.

\item Большасць асяроддзяў, у якіх пішацца код, дазваляе выкарыстоўванне каманд git унутры сябе, але я рэкамендую самастойна кантраляваць свой рэпазіторый.

\end{enumerate}

\end{document}