\documentclass[12pt, a4paper]{extarticle}
\pagenumbering{gobble}
\usepackage{geometry}
 \geometry{
 left=20mm,
 right=20mm,
 top=20mm,
 bottom=20mm,
 }

\usepackage{titlesec}
\titleformat{\section}[block]{\Large\bfseries\filcenter}{}{1em}{}
\titleformat{\subsection}[block]{\large\bfseries\filcenter}{}{1em}{}
\usepackage[utf8]{inputenc}
\usepackage[english, russian]{babel}

\usepackage[normalem]{ulem}
\usepackage{soul}
\usepackage{verbatim}
\usepackage{tabularx}
\usepackage{tabulary}
\usepackage{hyperref}
\usepackage{setspace}
\usepackage{amssymb,amsfonts,amsmath,cite,enumerate,float,indentfirst}
\usepackage{mathrsfs}
\usepackage{comment}
\usepackage{verbatim}
\usepackage{mathtools}
\usepackage{caption}
\usepackage{wrapfig}
\usepackage{mathtext} %cyrillic text in math mode, not recommended
\usepackage{amsthm}

\hypersetup{				
	unicode=true,           
	pdfkeywords={keyword1} {key2} {key3},
	colorlinks=true, 
	urlcolor=blue
}

\captionsetup[table]{justification=centering, singlelinecheck=false, font=large}
\makeatletter
\renewcommand{\fnum@table}{}
\makeatother

\newcommand{\formQA}[2]{%
	\noindent \textbf{Q:} #1 \\
	\textbf{A:} #2
}

\setlength{\parindent}{1.3em}
\setlength{\parskip}{10pt}
\usepackage{setspace}
\usepackage{enumitem}
\setlist[itemize]{topsep=0pt,partopsep=1ex,parsep=1ex}

\begin{document}

\section{git з github для імбрыкаў}

У гэтым доку будуць інструкцыя па наладжванню акаўнта на \href{https://github.com/}{github.com}, паслядоўнасць крокаў для стварэння там свайго рэпазіторыя, унутры якога можна карыстацца сродкамі git, а таксама базавыя каманды git і іншыя карысныя штукі. Для таго, каб усталяваць git на свой кампукцер, ёсць \href{https://git-scm.com/book/en/v2/Getting-Started-Installing-Git}{старонка} з інструкцыяй для кожнай аперацыйнай сістэмы.

На самай справе можна закрыць гэты док і знайсці нешта ў інтэрнэце, напрыклад \href{https://habr.com/ru/post/125799/}{гэта} (рэкамендую чытаць разам з каментамі), \href{https://www.tutorialspoint.com/git/index.htm}{гэта} ці \href{https://githowto.com}{гэта}.

\subsection{Акаўнт на github.com}

\begin{itemize}

\item Ствараем акаўнт: заходзім на сайт, клікаем на адну з кнопак \textbf{Sign up}, выконваем усе, што просяць.

\item Пасля стварэння акаўнта github.com/<username> будзе дамашняй старонкай.

\item Для бяспечнага карыстання рэкамендуецца наладзіць злучэнне з github па ssh. Ёсць добрая \href{https://help.github.com/en/articles/generating-a-new-ssh-key-and-adding-it-to-the-ssh-agent}{інструкцыя} для ўсіх асноўных аперацыйных сістэм. Каб праверыць, што злучэнне наладжана, трэба выканаць адну каманду ў кансолі, чытаць \href{https://help.github.com/en/articles/testing-your-ssh-connection}{тут}.

\end{itemize}

\subsection{git lfs}

Адны класныя дзядзечкі прыдумалі сродак, каб не мяшаць код і здаровыя і не вельмі бінарныя файлы ў адну кучу (то-бок можна падцягнуць з сервераў толькі код калі патрэбы ў астатнім няма ці трэба пампаваць вельмі шмат, а інтэрнэт не дазваляе па адной з прычын). Вельмі проста ўсталяваць, вельмі проста карыстацца, абы толькі не лагала. .-.

Інструкцыя па ўсталёўцы і наладжванні \href{https://github.com/git-lfs/git-lfs}{тут}. Толькі рэкамендую каманду \textbf{git lfs install --skip-smudge} замест \textbf{git lfs install}, каб можна было самастойна кіраваць зацягваннем вялікіх файлаў. 

\end{document}