\documentclass[12pt, a4paper]{extarticle}
\usepackage{import}

\subimport{../common/}{preamble}
\graphicspath{{images/}}

\begin{document}
	\section{\LaTeX (не латэкс)  для навічкоў}
	
	У гэтым дакуменце будзе дапамога па першых кроках у \LaTeX. Па большай частцы дакумент будзе складацца са спасылак на годныя штукі. 
	
	\subsection{Усталёўка}
	
	\href{https://www.latex-project.org/get/}{Тут} можна знайсці спосаб усталяваць нейкі з дыстрыбутываў на любую аперацыйную сістэму. Рэкамендую ставіць TeX Live. Калі карыстаешся Ubuntu або любым іншым лінуксам, які падтрымлівае apt, Tex Live можна ўсталяваць праз яго. Пасля таго, як LaTeX усталяваны, варта вызначыцца з тым, дзе рэдагаваць tex-дакументы. Ёсць спецыяльныя рэдактары, напрыклад, кросплатформены \href{https://www.texstudio.org}{texstudio}, які дазваляе кампіляваць дакументы ўнутры сябе. Але можна карыстацца любым тэкставым рэдактарам і пасля ў кансолі збіраць pdf-файл. Для прасунутых рэдактараў (Atom, VS Code, Sublime Text, Vim~($\wedge$\noindent\rule{0.3cm}{0.4pt}$\wedge$)) можна дадаць плагін, які спрашчае рэдагаванне tex-дакументаў. 
	
	Таксама можна карыстацца анлайн кампілятарамі, такімі, як \href{https://www.overleaf.com/project}{Overleaf}, \href{https://latexbase.com/}{LaTeX base}. Раней яшчэ быў Sharelatex, але яго аб'ядналі з Overleaf. З дапамогай анлайн-рэдактараў можна пісаць код разам з кентушкамі, як у гуглдоку, тока норм. Шмат плюсаў, можна нават воблака далучыць (за асобную плату), але калі інтэрнэт будзе зусім неоч, то будуць праблемы. Але часам анлайн-рэдактар выручае.
		
	\subsection{Хуткі старт}
		
	Неабавязкова нешта разумець, каб ствараць дакументы з дапамогай \LaTeX. Калі не патрабуюць нейкі стандарт афармлення, можна вельмі хутка навучыцца кляпаць докі з формуламі і тэкстам. Для гэтага трэба скапіяваць сабе прэамбулу (усё, што пішацца перад \verb|\begin{document}|), а пасля ўнутры акружэння document (акружэннне "--- гэта штука, якая задаецца з дапамогай тэгападобнай канструкцыі \verb|\begin{env} ... \end{env}|) пісаць свой тэкст, галоўнае, каб прэамбула змяшчала ў сабе ўсе неабходныя пакеты.
		
	Лепшая \href{http://web.ift.uib.no/Teori/KURS/WRK/TeX/symALL.html}{старонка} з матэматычнымі сімваламі, якую ведаю. Калі ў ёй не знайшоўся патрэбны сімвал, заўсёды можна нагугліць, але такое здараецца не вельмі часта. Калі захочацца прасунуцца далей у напісанні докаў, то можна альбо самастойна разбірацца, альбо самастойна з дапамогай спасылак, курсаў і туторыялаў, якія можна будзе знайсці далей ў гэтым доку.
	
	Матэматычныя формулы можна набіраць \href{http://mathurl.com}{анлайн}.
	
	Існуюць так званыя cheatsheets, у якіх зручна падглядаць базавыя штукі: вось \href{http://wch.github.io/latexsheet/latexsheet.pdf}{гэта} крутая. Я яшчэ думаў сваю каравую зрабіць, але пачаў і не дарабіў. :(
	
	\subsection{Яшчэ адзін спосаб не працягваць чытаць гэты док}
	
	Калі хочацца сур'ёзна разабрацца з LaTeX, можна пайсці на \href{https://www.coursera.org/learn/latex}{анлайн-курс} і паспрабаваць здаць. Лекцыі там даволі норм, заданні тож нічо, але я ў выніку апошняе заданне не здаў.~.-. Але ўсё адно рэкамендую. 
	
	Ёсць стандартныя спосабы праз усе гэтыя simple guide: \href{https://www.latex-tutorial.com/tutorials/}{раз} і \href{https://www.overleaf.com/learn/latex/Tutorials}{два}. Адкрываеце рэдактар, пішаце і паціху вучыцеся. Далей ужо ў інтэрнэце амаль усё ёсць. Заадно навучыце мяне ўстаўляць спасылкі з кірыліцай і працэнтамі. .\noindent\rule{0.4cm}{0.4pt}.
	
	\subsection{Прыклады}
	
	У якасці добрых прыкладаў звычайна можна браць здаравенныя аформленыя праекты, такія як курсач ці рэферат з кучай патрабаванняў. Калі той, хто піша код, будзе дадаваць каментары да далучэння пакетаў, будзе зусім цудоўна, бо там нельга глянуць, што робіць код.
	
	Напрыклад, ёсць \href{https://www.dropbox.com/sh/jjbg55273dsf5d4/AADXAADk5ZLBNx66fNPCiywca?dl=0}{рэферат} (пакуль свае чоткія зыходнікі на гітхуб не перавёз), аформлены па стандарце кандыдатскіх дысертацый (там можна нават крыху пачытаць пра \LaTeX, хехе). Там адрэдагавана амаль усё, што можна, і гэтыя параметры можна самастойна пакруціць і глянуць, як што працуе.
	
	Ёсць яшчэ здаравенны \href{https://www.dropbox.com/sh/0a4ffafuo0t8wmg/AABbUPWtXs9Y6G2veDsqKJdNa?dl=0}{дыплом} (бардак з файламі не раю паўтараць), дзе я таксама замарочыўся з афармленнем, але ў адрозненні ад рэферата там ёсць куча малюнкаў графаў (не самыя якасныя ў плане кода, канешне, але мне хапіла).
	
	Калі-небудзь у мяне з'явіцца вольны час (не), і я сваё дабро перанясу ў чоткую рэпку, і там будзе шмат усяго, акрамя, напэўна, самаробных графікаў з дапамогай pgf/tikz, таму што неяк не даводзілася, але прыклады ёсць \href{http://www.texample.net/tikz/}{тут}.
	
	У інтэрнэтах таксама можна пашукаць і знайсці карысныя штукі: \href{https://www.sharelatex.com/templates}{такая} і \href{http://www.latextemplates.com}{такая}. Калі трэба больш, у пошуку абавязкова знойдзецца.
	
	\subsection{Табліцы}
	
	Ёсць якасны \href{https://www.overleaf.com/learn/latex/tables}{док}, які можна дачытаць і патыркаць у спасылкі, каб яшчэ прашарыцца. Рэсурсы для стварэння табліц анлайн (робіш таблічку, прастаўляеш значэнні, а сайцец табе код падсуне, які трэба проста забраць сабе ў док): \href{http://www.tablesgenerator.com}{першы} і \href{https://truben.no/table/}{другі}.
	
	\subsection{Прэзентацыі}
	
	З дапамогай LaTeX можна рабіць крутыя прэзентацыі, для якіх можна ўжываць набраны тэкст працы, калі трэба нешта абараніць. Я сам не вельмі крута ўмею ў прэзентацыі, але існуе куча докаў, дзе можна чытануць і разабрацца. Але, як звычайна, лепш разок запарыцца, і потым ужо ўмець, як баця. На coursera ёсць блок лекцый пра прэзентацыі, таксама можна глянуць літаратурку: \href{http://ctan.uni-altai.ru/macros/latex/contrib/beamer/doc/beameruserguide.pdf}{здаравенны гайд}, \href{http://www.tug.org/pracjourn/2005-4/mertz/mertz.pdf}{beamer у прыкладах}, \href{https://en.wikibooks.org/wiki/LaTeX/Presentations}{дока на нейкую вікі}. Тэмы афармлення прэзентацый браць \href{https://hartwork.org/beamer-theme-matrix/}{тут}.
	
	Мая першая і пакуль адзіная \href{https://www.dropbox.com/sh/8wi5utnk43rpede/AAAHOtNlj1Nyz7vfbseezjpFa?dl=0}{tex-прэзентацыя}.
	
	\bigskip

	\textit{Калі пастаянна робіш докі ў \LaTeX, хутка запамінаецца, што дзе і як рэдагуецца, а калі і не, то ведаеш, дзе ўжо рабіў і лезеш туды глядзець. Але, каб не разлічваць на памяць, спасылкі на форумы з адказамі на нетрывіяльныя штукі раю недзе захоўваць (закладкі, як варыянт).}
\end{document}