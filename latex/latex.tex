\usepackage{geometry}
\usepackage{titlesec}
\usepackage[utf8]{inputenc}
\usepackage[english, russian]{babel}
\usepackage[normalem]{ulem}
\usepackage{soul}
\usepackage{verbatim}
\usepackage{tabularx}
\usepackage{tabulary}
\usepackage{hyperref}
\usepackage{setspace}
\usepackage{amssymb,amsfonts,amsmath,cite,enumerate,float,indentfirst}
\usepackage{mathrsfs}
\usepackage{comment}
\usepackage{verbatim}
\usepackage{mathtools}
\usepackage{caption}
\usepackage{wrapfig}
\usepackage{mathtext} %cyrillic text in math mode, not recommended
\usepackage{amsthm}
\usepackage{graphicx}
\usepackage{setspace}
\usepackage{enumitem}
\usepackage[table]{xcolor}

\geometry{left=20mm, right=10mm, top=20mm, bottom=20mm}

\setlist[itemize]{topsep=0pt,partopsep=1ex,parsep=1ex}

\titleformat{\section}[block]{\Large\bfseries\filcenter}{}{1em}{}
\titleformat{\subsection}[block]{\large\bfseries\filcenter}{}{1em}{}

\hypersetup{				
	unicode=true,           
	pdfkeywords={keyword1} {key2} {key3},
	colorlinks=true, 
	urlcolor=blue,
	linkcolor=black
}

\setlength{\parindent}{1.3em}
\setlength{\parskip}{10pt}
\pagenumbering{gobble}

% figure captions settings
\captionsetup[figure]{labelfont=bf, justification=centering}
\renewcommand\thefigure{\thesection.\arabic{figure}}  
\makeatletter
\renewcommand{\fnum@figure}{Малюнак \thefigure}
\makeatother

% table captions settings
\captionsetup[table]{justification=centering, singlelinecheck=false, font=large}
\makeatletter
\renewcommand{\fnum@table}{}
\makeatother
\renewcommand\thetable{\thesection.\arabic{table}}  
\makeatletter
\renewcommand{\fnum@table}{Табліца \thetable}
\makeatother

\newcommand{\formQA}[2]{%
	\noindent \textbf{Q:} #1 \\
	\textbf{A:} #2
}


\begin{document}
	\section{\LaTeX (не латэкс)  для навічкоў}
	
	У гэтым дакуменце будзе дапамога для першых крокаў у \LaTeX. Па большай частцы дакумент будзе складацца са спасылак на годныя штукі. 
	
	\subsection{Усталёўка}
	
	\href{https://www.latex-project.org/get/}{Тут} можна знайсці спосаб усталяваць нейкі з дыстрыбутываў на любую аперацыйную сістэму. Рэкамендую ставіць TeX Live. Калі карыстаешся Ubuntu або любым іншым лінуксам, які падтрымлівае apt, Tex Live можа ўсталяваць праз яго. Пасля таго, як \LaTeX усталяваны, варта вызначыцца з тым, дзе рэдагаваць tex-дакументы. Ёсць спецыяльныя рэдактары, напрыклад, кросплатформены \href{https://www.texstudio.org}{texstudio}, які дазваляе кампіляваць дакументы ўнутры сябе. Але можна карыстацца любым тэкставым рэдактарам і пасля ў кансолі збіраць pdf файл. Для прасунутых рэдактараў (Atom, VS Code, Sublime Text, Vim~($\wedge$\noindent\rule{0.3cm}{0.4pt}$\wedge$)) можна дадаць плагін, які спрашчае рэдагаванне tex дакументаў. 
	
	Таксама можна карыстацца анлайн кампілятарамі, такімі, як \href{https://www.overleaf.com/project}{Overleaf}, \href{https://latexbase.com/}{LaTeX base}. Раней яшчэ быў Sharelatex, але яго аб'ядналі з Overleaf. З дапамогай анлайн-рэдактараў можна пісаць код разам з кентушкамі, як у гуглдоку, тока норм. Шмат плюсаў, можна нават воблака далучыць (за асобную плату), але калі інтэрнэт будзе зусім неоч, то будуць праблемы. Але часам выручае.
		
	\subsection{Хуткі старт}
		
	Неабавязкова нешта разумець, каб ствараць дакументы з дапамогай \LaTeX. Калі не патрабуюць нейкі стандарт афармлення, можна вельмі хутка навучыцца кляпаць докі з формуламі і тэкстам. Для гэтага трэба скапіяваць сабе прэамбулу (усё, што пішацца перад \verb|\begin{document}|), а пасля ўнутры акружэння document (акружэннне "--- гэта штука, якая задаецца з дапамогай тэгападобнай канструкцыі \verb|\begin{env} ... \end{env}|) пісаць свой тэкст, галоўнае, каб прэамбула змяшчала ў сабе ўсе неабходныя пакеты.
		
	Лепшая \href{http://web.ift.uib.no/Teori/KURS/WRK/TeX/symALL.html}{старонка} з матэматычнымі сімваламі, якую ведаю. Калі ў ёй не знайшоўся патрэбны сімвал, заўсёды можна нагугліць, але такое здараецца не вельмі часта. Калі захочацца прасунуцца далей у напісанні докаў, то можна альбо самастойна разбірацца, альбо самастойна з дапамогай спасылак, курсаў і туторыялаў, якія можна будзе знайсці далей ў гэтым доку.
	
	\subsection{Яшчэ адзін спосаб не працягваць чытаць гэты док}
	
	Калі хочацца сур'ёзна разабрацца з LaTeX, можна пайсці на \href{https://www.coursera.org/learn/latex}{анлайн-курс} і паспрабаваць здаць. Лекцыі там даволі норм, заданні тож нічо, але я ў выніку апошняе заданне не здаў.~.-. Але ўсё адно рэкамендую. 
	
	Ёсць стандартныя спосабы праз усе гэтыя simple guide: \href{https://www.latex-tutorial.com/tutorials/}{раз} і \href{https://www.overleaf.com/learn/latex/Tutorials}{два}. Адкрываеце рэдактар, пішаце і паціху вучыцеся. Далей ужо ў інтэрнэце амаль усё ёсць. Заадно навучыце мяне ўстаўляць спасылкі з кірыліцай і працэнтамі. .\noindent\rule{0.4cm}{0.4pt}.
	
	
\end{document}